\begin{figure}[h]
\centering \includegraphics[scale=0.4]{images/php.png}
\caption{Php logo}
\end{figure}
\noindent PHP is an open source server-side scripting language designed for Web development to produce dynamic Web pages. It is one of the first developed server-side scripting languages to be embedded into an HTML source document rather than calling an external file to process data. The code is interpreted by a Web server with a PHP processor module which generates the resulting Web page. It also has evolved to include a command-line interface capability and can be used in standalone graphical applications.\\

\noindent PHP can be deployed on most Web servers and also as a standalone shell on almost every operating system and platform, free of charge. A competitor to Microsoft’s Active Server Pages (ASP) server-side script engine and similar languages, PHP is installed on more than 20 million Web sites and 1 million Web servers. Notable software that uses PHP includes Drupal, Joomla, MediaWiki, and WordPress. PHP is a general-purpose scripting language.\\

\noindent It is especially suited to server-side web development where PHP generally runs on a web server. Any PHP code in a requested file is executed by the PHP runtime, usually to create dynamic web page content or dynamic images used on Web sites or elsewhere. It can also be used for command-line scripting and client-side graphical user interface (GUI) applications. PHP can be deployed on most Web servers, many operating systems.
\subsection{Features of PHP}
\begin{itemize}
\item Http Authentication
\item Cookies and Sessions
\item Connection Handling
\item Designer-friendly 
\item Cross platform Compatibility 
\item Loosely typed Language
\item Open Source
\item Easy code
\end{itemize}


