\section{Introduction to Languages,IDE’s,Tools and Technologies used for Implementation}
	
\subsection{Technologies used}

\subsubsection{Firebase}

Firebase helps you build better mobile apps and grow your business.

Firebase is a mobile platform from Google offering a number of different features that you can pick ‘n mix from. Specifically, these features revolve around cloud services, allowing users to save and retrieve data to be accessed from any device or browser. This can be useful for such things as cloud messaging, hosting, crash reporting, notifications, analytics and even earning money through AdMob.

It works with Android apps, iOS apps and web apps and best of all: it’s free!

\begin{itemize}

\item \textbf{Setting up a project}

Before you can do anything with Firebase, you first need to create an account. You can do this over at firebase.google.com.

\begin{figure}[ht]
\centering
\includegraphics[scale=0.20]{images/Pdf2.png}
\caption{Firebase Console}
\end{figure}

 

\item \textbf{Firebase Products}

\begin{figure}[ht]
\centering
\includegraphics[scale=0.4]{images/Pdf1.png}
\caption{Firebase Products}
\end{figure}

\begin{itemize}

\item \textbf{Firebase Authentication}

Firebase Authentication aims to make building secure authentication systems easy, while improving the sign-in and onboarding experience for end users. It provides an end-to-end identity solution, supporting email and password accounts, phone auth, and Google, Twitter, Facebook, and GitHub login, and more.

Most apps need to know the identity of a user. Knowing a user's identity allows an app to securely save user data in the cloud and provide the same personalized experience across all of the user's devices.
Firebase Authentication provides backend services, easy-to-use SDKs, and ready-made UI libraries to authenticate users to your app. It supports authentication using passwords, phone numbers, popular federated identity providers like Google, Facebook and Twitter, and more.

Firebase Authentication integrates tightly with other Firebase services, and it leverages industry standards like OAuth 2.0 and OpenID Connect, so it can be easily integrated with your custom backend.

\item \textbf{Firebase Realtime Database}

Store and sync data with our NoSQL cloud database. Data is synced across all clients in realtime, and remains available when your app goes offline.

The Firebase Realtime Database is a cloud-hosted database. Data is stored as JSON and synchronized in realtime to every connected client. When you build cross-platform apps with our iOS, Android, and JavaScript SDKs, all of your clients share one Realtime Database instance and automatically receive updates with the newest data.


\item \textbf{Cloud Storage}

Cloud Storage is built for app developers who need to store and serve user-generated content, such as photos or videos.

Cloud Storage for Firebase is a powerful, simple, and cost-effective object storage service built for Google scale. The Firebase SDKs for Cloud Storage add Google security to file uploads and downloads for your Firebase apps, regardless of network quality. You can use our SDKs to store images, audio, video, or other user-generated content. On the server, you can use Google Cloud Storage, to access the same files.
	\end{itemize}
	\end{itemize}
	
\subsubsection{Web Technology}


Web technology is a broad term for the work involved in developing a web site for the Internet (World Wide Web) or an intranet (a private network). Web development can range from developing the simplest static single page of plain text to the most complex web-based internet applications (or just 'web apps') electronic businesses, and social network services. A more comprehensive list of tasks to which web development commonly refers, may include web engineering, web design, web content development, client liaison, client-side/server-side scripting, web server and network security configuration, and e-commerce development. Among web professionals, "web development" usually refers to the main non-design aspects of building web sites: writing markup and coding. 

\begin{figure}[ht]
\centering
\includegraphics[scale=0.48]{images/WebTechnology.jpg}
\caption{Web Development Anatomy}
\label{fig:Anatomy}
\end{figure}

Most recently Web development has come to mean the creation of content management systems or CMS. These CMS can be made from scratch, proprietary or open source. In broad terms the CMS acts as middleware between the database and the user through the browser. A principle benefit of a CMS is that it allows non-technical people to make changes to their web site without having technical knowledge.

For larger organizations and businesses, web development teams can consist of hundreds of people (web developers) and follow standard methods like Agile methodologies while developing websites. Smaller organizations may only require a single permanent or contracting developer, or secondary assignment to related job positions such as a graphic designer or information systems technician. Web development may be a collaborative effort between departments rather than the domain of a designated department. There are three kinds of web developer specialization: front-end developer, back-end developer, and full-stack developer. Front-end developers deal with the layout and visuals of a website, while back-end developers deal with the functionality of a website. Back-end developers will program in the functions of a website that will collect data.
Since the commercialization of the web, web development has been a growing industry. The growth of this industry is being driven by businesses wishing to use their website to sell products and services to customers.[1]

There is open source software for web development like BerkeleyDB, GlassFish, LAMP (Linux, Apache, MySQL, PHP) stack and Perl/Plack. This has kept the cost of learning web development to a minimum. Another contributing factor to the growth of the industry has been the rise of easy-to-use WYSIWYG web-development software, such as Adobe Dreamweaver, BlueGriffon and Microsoft Visual Studio. Knowledge of HyperText Markup Language (HTML) or of programming languages is still required to use such software, but the basics can be learned and implemented quickly with the help of help files, technical books, internet tutorials, or face-to-face training.



\begin{itemize}

\item \textbf{Activity Lifecycle}
Activities in the system are managed as an activity stack. When a new activity
is started, it is placed on the top of the stack and becomes the running activitythe previous activity always remains below it in the stack, and will not come to
the foreground again until the new activity exits. An activity has essentially four
states:
If an activity in the foreground of the screen (at the top of the stack), it is
active or running.

If an activity has lost focus but is still visible (that is, a new non-full-sized or
transparent activity has focus on top of your activity), it is paused. A paused
activity is completely alive (it maintains all state and member information and
remains attached to the window manager), but can be killed by the system in
extreme low memory situations.


\end{itemize}

\subsection{Language used}
\subsubsection{JAVASCRIPT}

JavaScript is a high-level, interpreted programming language. It is a language which is also characterized as dynamic, weakly typed, prototype-based and multi-paradigm.

Alongside HTML and CSS, JavaScript is one of the three core technologies of the World Wide Web. JavaScript enables interactive web pages and thus is an essential part of web applications. The vast majority of websites use it, and all major web browsers have a dedicated JavaScript engine to execute it.

As a multi-paradigm language, JavaScript supports event-driven, functional, and imperative (including object-oriented and prototype-based) programming styles. It has an API for working with text, arrays, dates, regular expressions, and basic manipulation of the DOM, but the language itself does not include any I/O, such as networking, storage, or graphics facilities, relying for these upon the host environment in which it is embedded.

Initially only implemented client-side in web browsers, JavaScript engines are now embedded in many other types of host software, including server-side in web servers and databases, and in non-web programs such as word processors and PDF software, and in runtime environments that make JavaScript available for writing mobile and desktop applications, including desktop widgets.

Although there are strong outward similarities between JavaScript and Java, including language name, syntax, and respective standard libraries, the two languages are distinct and differ greatly in design; JavaScript was influenced by programming languages such as Self and Scheme.

The characteristics and features of java are as follows :

\begin{itemize}
	\item \textbf{Browser support}
To access flash content, you need to install flash plugin in your browser. But to use javascript, you don't have to use any plugin at all. This is because all browsers have accepted javascript as a scripting language for them and provides integrated support for it. All you need to do is to handle some of the tasks that are dependent on DOM (Document Object Model) of different browsers properly.


\item \textbf{Can be used on client side as well as on server side}
As javascript has access to Document object model of browser, you can actually change the structure of web pages at runtime. Due to this, javascript can be used to add different effects to webpages. On the other hand, javascript could be used on the server side as well. For example, in Alfresco which is a popular open source enterprise content management system, javascript is used in creating webscripts. This makes adding custom tasks to alfresco quite simple.

\begin{itemize}
	\item Functional programming language
\item Support for objects

\end{itemize}

\item \textbf{Secure}
Secure Javascript is Secure Language because of its many features it enables to
develop virus-free, tamper-free systems. Authentication techniques are
based on public-key encryption.

\item \textbf{Robust}
Robust Javascript was created as a strongly typed language. Data type issues
and problems are resolved at compile-time, and implicit casts of a variable
from one type to another are not allowed.

\item \textbf{Architectural Neural}
Architecture neutral It is not easy to write an application that can be used
on Windows , UNIX and a Macintosh. And its getting more complicated
with the move of windows to non Intel CPU architectures.
Javascript takes a diffierent approach. 
\item \textbf{Portable}
Portable Javascript code is portable. It was an important design goal of Javascript that it
be portable so that as new architectures(due to hardware, operating system,
or both)

\item \textbf{High performance}
High performance For all but the simplest or most infrequently used appli-cations,
performance is always a consideration for most applications, including
21graphics-intensive ones such as are commonly found on the world wide web,
the performance of javascript is more than adequate.

\item \textbf{ Generating HTML on the fly}
This Table of Contents is dynamically expandable. To view all subsections in a section, you can click on the white arrow  corresponding to that section. To hide subsections, click on the arrow .

Every time you click on the arrows, the browser generates and displays new HTML code in the left frame. Thanks to JavaScript, this operation is performed on the client machine, and therefore you don't have to wait while the information goes back and forth between your browser and the Web server.




\end{itemize}

\subsubsection{HTML}
Hypertext Markup Language (HTML) is the standard markup language for creating web pages and web applications. With Cascading Style Sheets (CSS) and JavaScript, it forms a triad of cornerstone technologies for the World Wide Web.[4]

Web browsers receive HTML documents from a web server or from local storage and render the documents into multimedia web pages. HTML describes the structure of a web page semantically and originally included cues for the appearance of the document.

HTML elements are the building blocks of HTML pages. With HTML constructs, images and other objects such as interactive forms may be embedded into the rendered page. HTML provides a means to create structured documents by denoting structural semantics for text such as headings, paragraphs, lists, links, quotes and other items. HTML elements are delineated by tags, written using angle brackets. Tags such as <img /> and <input /> directly introduce content into the page. Other tags such as <p> surround and provide information about document text and may include other tags as sub-elements. Browsers do not display the HTML tags, but use them to interpret the content of the page.

\subsubsection{CSS}
Cascading Style Sheets (CSS) is a style sheet language used for describing the presentation of a document written in a markup language like HTML.[1] CSS is a cornerstone technology of the World Wide Web, alongside HTML and JavaScript.[2]

CSS is designed to enable the separation of presentation and content, including layout, colors, and fonts.[3] This separation can improve content accessibility, provide more flexibility and control in the specification of presentation characteristics, enable multiple web pages to share formatting by specifying the relevant CSS in a separate .css file, and reduce complexity and repetition in the structural content.

Separation of formatting and content also makes it feasible to present the same markup page in different styles for different rendering methods, such as on-screen, in print, by voice (via speech-based browser or screen reader), and on Braille-based tactile devices. CSS also has rules for alternate formatting if the content is accessed on a mobile device.

\subsection{IDE used}
\subsubsection{Visual Studio}

\begin{figure}[ht]
\centering
\includegraphics[scale=0.90]{images/VisualStudio.png}
\caption{Visual Studio}
\end{figure}

Microsoft Visual Studio is an integrated development environment (IDE) from Microsoft. It is used to develop computer programs, as well as web sites, web apps, web services and mobile apps. Visual Studio uses Microsoft software development platforms such as Windows API, Windows Forms, Windows Presentation Foundation, Windows Store and Microsoft Silverlight. It can produce both native code and managed code.

Visual Studio includes a code editor supporting IntelliSense (the code completion component) as well as code refactoring. The integrated debugger works both as a source-level debugger and a machine-level debugger. Other built-in tools include a code profiler, forms designer for building GUI applications, web designer, class designer, and database schema designer. It accepts plug-ins that enhance the functionality at almost every level—including adding support for source control systems (like Subversion) and adding new toolsets like editors and visual designers for domain-specific languages or toolsets for other aspects of the software development lifecycle (like the Team Foundation Server client: Team Explorer).





\begin{itemize}
\item \textbf{System Requirements}
Web application development on either of the following operating systems −
\vskip 0.1in

Microsoft Windows 10/8/7/Vista/2003 (32 or 64-bit)
\\Mac OS X 10.8.5 or higher, up to 10.9 (Mavericks)
\\GNOME or KDE desktop
\vskip 0.1in

\end{itemize}

\subsection{Introduction to \LaTeX}
\begin{figure}[ht]
\centering
\includegraphics[scale=0.2]{images/latex.png}
\caption{\LaTeX{} Logo}
\end{figure}
\hspace{-1.8em} \LaTeX{}, I had never heard about this term before doing this project,
but when I came to know about it's features, it is just excellent. 
\LaTeX (pronounced /ˈleɪtɛk/, /ˈleɪtɛx/, /ˈlɑːtɛx/, or /ˈlɑːtɛk/) is a 
document markup language and document preparation system for the \TeX{} 
typesetting  program. Within the typesetting system, its name is styled 
as \LaTeX.

\hspace{-1.8em} Within the typesetting system, its name is styled as \LaTeX. The term 
\LaTeX{} refers only to the language in which documents are written, 
not to the editor used to write those documents. In order to create a 
document in \LaTeX, a .tex file must be created using some form of text 
editor. While most text editors can be used to create a \LaTeX{} document, 
a number of editors have been created specifically for working with \LaTeX.\\

\noindent\LaTeX{} is most widely used by mathematicians, scientists, 
engineers, philosophers, linguists, economists and other scholars in 
academia. As a primary or intermediate format, e.g., translating DocBook 
and other XML-based formats to PDF, \LaTeX{} is used because of the 
high quality of typesetting achievable by \TeX. The typesetting system 
offers programmable desktop publishing features and extensive facilities 
for automating most aspects of typesetting and desktop publishing, 
including numbering and cross-referencing, tables and figures, 
page layout and bibliographies.\\

\noindent\LaTeX{} is intended to provide a high-level language that
accesses the power of \TeX. \LaTeX{} essentially comprises a
collection of \TeX{} macros and a program to process \LaTeX documents. 
Because the \TeX{} formatting commands are very low-level, it is usually 
much simpler for end-users to use \LaTeX{}.


\subsubsection{Typesetting}
\LaTeX{} is based on the idea that authors should be able to focus on 
the content of what they are writing without being distracted by its 
visual presentation. in preparing a \LaTeX{} document, the author 
specifies the logical structure using familiar concepts such as 
chapter, section, table, figure, etc., and lets the \LaTeX{} system 
worry about the presentation of these structures. it therefore 
encourages the separation of layout from content while still allowing 
manual typesetting adjustments where needed. 

\begin{verbatim}
\documentclass[12pt]{article}
\usepackage{amsmath}
\title{\LaTeX}
\begin{document}
  \maketitle 
  \LaTeX{} is a document preparation system 
  for the \TeX{} typesetting program.
   \par 
   $E=mc^2$
\end{document}
\end{verbatim}

\subsubsection{Installing \LaTeX{} on System}
Installation of \LaTeX{} on personal system is quite easy. As i have used \LaTeX{} on Ubuntu 13.04 so i am discussing the installation steps for Ubuntu 13.04 here:
\begin{itemize}
\item Go to terminal and type\\\\
\textit{sudo apt-get install texlive-full}
\item Your Latex will be installed on your system and you can check for manual page by typing.\\\\
\textit{man latex}\\

in terminal which gives manual for latex command.
\item To do very next step now one should stick this to mind that the document which one is going to produce is written in any type of editor whether it may be your most common usable editor Gedit or you can use vim by installing first vim into your system using command.\\\\
\textit{sudo apt-get install vim}
\item After you have written your document it is to be embedded with some set of commands that Latex uses so as to give a structure to your document. Note that whenever you wish your document to be looked into some other style just change these set of commands.
\item When you have done all these things save your piece of code with .tex format say test.tex. Go to terminal and type\\\\
\textit{latex path of the file test.tex Or pdflatex path of the file test.tex\\ eg: pdflatex test.tex}\\
for producing pdf file simultaneously.\\
After compiling it type command\\\\
\textit{evince filename.pdf\\ eg: evince test.pdf}\\
To see output pdf file. 
\end{itemize}
\subsubsection{Pdfscreen \LaTeX{}}
There are some packages that can help to have unified document using \LaTeX{}. Example of such a package is pdfscreen that let the user view it’s document in two forms-print and screen. Print for hard copy and screen for viewing your document on screen. Download this package from www.ctan.org/tex-archive/macros/latex/contrib/pdfscreen/.\\
Then install it using above mention method.\\

\noindent To test it the test code is given below:-\\
Just changing print to screen gives an entirely different view. But for working of pdfscreen another package required are comment and fancybox.\\

\noindent The fancybox package provides several different styles of boxes for framing and rotating content in your document. Fancybox provides commands that produce square-cornered boxes with single or double lines, boxes with shadows, and round-cornered boxes with normal or bold lines. You can box mathematics, floats, center, flushleft, and flushright, lists, and pages.\\
 	
\noindent Whereas comments package selectively include/excludes portions of text. The comment package allows you to declare areas of a document to be included or excluded. One need to make these declarations in the preamble of your file. The package uses a method for exclusion that is pretty robust, and can cope with ill-formed bunches of text.\\

\noindent So these extra packages needed to be installed on system for the proper working of pdfscreen package.


	\section{Coding standards of Language used 
}

\subsection{Coding standards for Javascript }
\begin{itemize}

\item Indentation with tabs.
\item No whitespace at the end of line or on blank lines.
\item Lines should usually be no longer than 80 characters, and should not exceed 100 (counting tabs as 4 spaces). This is a “soft” rule, but long lines generally indicate unreadable or disorganized code.
\item If/else/for/while/try blocks should always use braces, and always go on multiple lines.
\item Unary special-character operators (e.g., ++, --) must not have space next to their operand.
\item Any, and; must not have preceding space.
\item Any; used as a statement terminator must be at the end of the line.
\item Any: after a property name in an object definition must not have preceding space.
\item The? and: in a ternary conditional must have space on both sides.
\item No filler spaces in empty constructs (e.g., {}, [], fn ()).
\item There should be a new line at the end of each file.
\item Any! Negation operator should have a following space.
\end{itemize}
\subsection{Coding standards for html and css}
\begin{itemize}
\item Validation:-All HTML pages should be verified against the W3C validator to ensure that the mark-up is well formed. This in and of itself is not directly indicative of good code, but it helps to weed out problems that are able to be tested via automation. It is no substitute for manual code review.
\item Self-closing Elements:-All tags must be properly closed. For tags that can wrap nodes such as text or other elements, termination is a trivial enough task. For tags that are self-closing, the forward slash should have exactly one space preceding it.
\item Attributes and Tags:-All tags and attributes must be written in lowercase. Additionally, attribute values should be lowercase when the purpose of the text therein is only to be interpreted by machines. For instances in which the data needs to be human readable, proper title capitalization should be followed.
\item Quotes:-According to the W3C specifications for XHTML, all attributes must have a value, and must use double- or single-quotes (source). The following are examples of proper and improper usage of quotes and attribute/value pairs.

\end{itemize}

	\section{GANTT chart
	}
	\begin{figure}[ht]
\centering
\includegraphics[scale=0.20]{images/GhanttChart.png}
\caption{\LaTeX{} Ghant chart}
\end{figure}
	
	\section{Testing Techniques and Test Plans
}
\subsection{Functionality Testing}
Test for – all the links in web pages, database connection, forms used in the web pages for
submitting or getting information from user, cookie testing.

\subsection{Usability testing}
Web site should be easy to use. Instructions should be provided clearly. Check if the provided
instructions are correct meaning whether they satisfy the purpose. Main menu should be
provided on each page. It should be consistent.

\subsection{Interface Testing}
The main interfaces are:
\\ i. Web server and application server interface
\\ ii. Application server and database server interface
Check if all the interactions between these servers are executed properly. Errors are handled
properly. If database or web server returns any error message for any query by application
server, then application server should catch and display these error messages appropriately to
users. Check what happens if user interrupts any transaction in-between? Check what happens
if connection to web server is reset in between?
\subsection{Compatibility Testing}
Compatibility of your web site is very important testing aspect. See which compatibility test is
to be executed:
\\i. Browser compatibility
\\ii. Operating system compatibility
\\iii. Mobile browsing
\\iv. Printing options
\subsection{Performance Testing}
Web application should sustain to heavy load. Web performance testing should include:
\\i. Web Load Testing
\\ii. Web Stress Testing

\subsection{Security Testing}
\begin{itemize}


\item Test by pasting internal URL directly into browser address bar without login.
Internal pages should not open.
\item If you are logged in using username and password and browsing internal pages,
then try changing URL options directly, i.e., If you are checking some publisher site
statistics with publisher site ID= 123. Try directly changing the URL site ID parameter
to different site ID which is not related to logged in user. Access should be denied for
this user to view others stats.
\item Try some invalid inputs in input fields like login username, password, input text
boxes. Check the system reaction on all invalid inputs.
iv. Web directories or files should not be accessible directly unless given download
option.
\end{itemize}

